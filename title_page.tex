\begin{center}
\bf{The Pennsylvania State University} \\
The Graduate School \\
College of Medicine Public Health Sciences \\

STATISTICAL MODELS FOR HIGH DIMENSIONAL SCREENING OF GENETIC AND EPIGENETIC EFFECTS

A Dissertation in \\
Biostatistics \\
by \\
Kirk Gosik \\
(c) 2017 Kirk Gosik \\

Doctor of Philosophy \\
February 2017 \\

\newpage
\pagenumbering{roman}
\setcounter{page}{2}
The dissertation of Kirk Gosik was reviewed and approved* by the following:

Rongling Wu \\
Distinguished Professor of Public Health Sciences and Statistics \\
Thesis Advisor, Chair of Committee

Vernon Chinchilli \\
Distinguished Professor and Chair of Public Health Sciences


Lan Kong \\
Associate Professor of Public Health Sciences

James Broach \\
Distinguished Professor and Chair of Biochemistry and Molecular Biology \\


\newpage

Abstract

Knowledge about how changes in gene expression are encoded by expression quantitative trait loci (eQTLs) is a key to construct the genotype-phenotype map for complex traits or diseases. Traditional eQTL mapping is to associate one transcript with a single marker at a time, thereby limiting our inference about a complete picture of the genetic architecture of gene expression. Here, I present innovative applications of variable selection approaches to systematically detect main effects and interaction effects among all possible loci on differentiation and function of gene expression and other phenotypes of interest. Forward-selection-based procedures were particularly implemented to tackle complex covariance structures of gene-gene interactions. Simulation studies were performed on each of the models to assess the computational properties of each model.  Applications of the models were also performed on real datasets.  The first was a reanalysis of a published genetic and genomic dataset collected in a mapping population of Caenorhabditis elegans, gaining new discoveries on the genetic origin of gene expression differentiation, which could not be detected by a traditional one-locus/one-transcript analysis approach.  The next dataset was of Mei Tree growth, analyzing the genetic control of the height and diameter during the developmental process.  The underlying genotypes and epistasis that impact the process of these developments were considered as candidates for the selection of the procedure.

\end{center}
